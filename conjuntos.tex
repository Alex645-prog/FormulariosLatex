\documentclass{formulario}

\begin{document}

\titulo{Conjuntos}
\subtitulo{Leyes básicas y Conteo}

    % ==========================================
    \section{Representación}
    % ==========================================
    Si $A$ y $B$ son dos conjuntos, y $U$ es el conjunto universo, entonces:
    
    \begin{multicols}{4}
        \eqn{Unión}{A \cup B}
        \eqn{Intersección}{A \cap B}
        \eqn{Diferencia}{A - B}
        \eqn{Diferencia simétrica:}{A \triangle B}
        \eqn{Complemento}{A' = A'}
        \eqn{Subconjunto}{A \subset B}
        \eqn{Cardinalidad}{|A| = \#A = Card |A|}
        \eqn{Potencia}{P(A)}
    \end{multicols}

    % ==========================================
    \section{Leyendas del Álgebra de conjuntos}
    % ==========================================
    
    \begin{multicols}{2}
        
        \begin{subequations}
            \eqnnumber{Leyes idempotentes}{
                A \cup A = A\\
                    A \cap A = A
            }
        \end{subequations}

        \begin{subequations}
        \eqnnumber{Leyes asociativas}{
                (A \cup B) \cup C &= A \cup (B \cup C)\\
                (A \cap B) \cap C &= A \cap (B \cap C)
        }
        \end{subequations}

        \begin{subequations}
            \eqnnumber{Leyes conmutativas}{
                    A \cup B &= B \cup A\\
                    A \cap B &= B \cap A
            }
        \end{subequations}

        \begin{subequations}
            \eqnnumber{Leyes distributivas}{
                A \cup (B \cap C) &= (A \cup B) \cap (A \cup C)\\
                A \cap (B \cup C) &= (A \cap B) \cup (A \cap C)
            }
        \end{subequations}

        \columnbreak 

        \begin{subequations}
            \eqnnumber{Leyes de identidad y absorción}{
                A \cup \phi &= A\\
                A \cap U &= A\\
                A \cup U &= U\\
                A \cap \phi &= \phi
            }
        \end{subequations}

        \eqnnumber{Ley involutiva}{
            (A')' &= A
        }

        \begin{subequations}
            \eqnnumber{Leyes del complementario}{
                A \cup A' &= U\\
                A \cap A' &= \phi \\
                U' &= \phi\\
                \phi' &= U
            }
        \end{subequations}
        
        \begin{subequations}
            \eqnnumber{Leyes de Morgan}{
                (A \cup B)' &= A' \cap B'\\
                (A \cap B)' &= A' \cup B'
            }
        \end{subequations}
  
    \end{multicols} 
    
    % ==========================================
    \section{Principios de Conteo}
    % ==========================================
    


        \eqn{Unión e Intersección (2 conjuntos)}{
            | A \cup B | = | A | + | B | - | A \cap B |
        }
        
        \eqn{Unión e Intersección (3 conjuntos)}{
            | A \cup B \cup C | &= | A | + | B | + | C |
             - |B \cap C| - |A \cap B| - |A \cap C|
             + |A \cap B \cap C |
        }
    
        \eqn{Diferencia}{
            | A - B | = | A \cup B | - | B | = | A | - | A \cap B |
        }
    
        \eqn{Diferencia simétrica}{
            | A \triangle B | = | A \cup B | - | A \cap B |
        }
        


\end{document}